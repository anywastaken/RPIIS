\documentclass[10pt,a4paper,twocolumn]{article}
\usepackage{graphicx} 
\usepackage{enumitem}
\usepackage[labelformat=empty]{caption}
\usepackage[english, russian]{babel}
\usepackage[utf8]{inputenc}
\usepackage[margin=1.5cm,top=1.5cm]{geometry}
\usepackage{import}
\usepackage{scn}
\usepackage[nottoc]{tocbibind} 
\setcounter{figure}{3}
\renewcommand{\tocbibname}{\normalsize \centering \textnormal{
References}}
\newcommand{\RomanNumeralCaps}[1]
{\MakeUppercase{\romannumeral #1}}
\date{}
\title{}
\setcounter{page}{234}
\begin{document}
\maketitle
taking into account the previously mentioned entities
and understanding how they relate to the current topic
of conversation. This makes the dialog with the system
more productive and natural, similar to talking to a
person who remembers all the details to solve a problem.
\vspace{0.2cm}
\par \RomanNumeralCaps{3}.Example of application of the results obtained
\vspace{0.2cm}
\par An example of realization of the proposed approach
is an intelligent geometry learning system consisting of
three components: a KB, a problem solver and a UI.
\par Below is a decomposition of the Geometry Intelligent
System Problem Solver. It consists of a search module---
agents that search for constructs in the KB and a computational problem solver, i.e., agents that implement algorithms for solving geometry problems.
\vspace{0.3cm}
\\ \textbf{Solver of intelligent system problems in geometry}
\begin{scnrelfromset}{\hspace{-0.6cm} decomposition of an abstract sc-agent}
\scnitem {\hspace{-0.65cm}Abstract non-atomic sc-agent search agent}
\scnitem{\hspace{-0.8cm} omputational Problem Solver}
\begin{scnrelfromset}{\hspace{-0.5cm}decomposition of an abstract sc-agent}
      \scnitem{\hspace{-0.65cm}Abstract sc-agent for interpreting
arithmetic expressions}
      \scnitem{\hspace{-0.65cm}Abstract non-atomic sc-agent for
interpreting logical rules}
\scnitem{\hspace{-0.65cm}Abstract sc-agent of constructing a
strategy for finding a solution to a
problem in width}
      \end{scnrelfromset}
\end{scnrelfromset}
Next, a decomposition of Abstract non-atomic sc-agent search is presented. It consists of a necessary set of scagents that can be used in solving specific problems. For example, if a theorem proving problem needs to
be solved, it is appropriate to use Abstract sc-agent for searching axioms of a given ontology or Abstract sc-agent for searching theorems of a given ontology. 
\vspace{0.45cm}
\\ \textbf{Abstract non-atomic sc-agent search agent}
\begin{scnrelfromset} {\hspace{-0.5cm}decomposition of an abstract sc-agent}
\scnitem {\hspace{-0.65cm}Abstract sc-agent for finding an annotation for a
given section}
\scnitem{\hspace{-0.8cm} Abstract sc-agent for searching axioms of a
given ontology}
\scnitem{\hspace{-0.8cm} Abstract sc-agent for searching theorems of a
given ontology}
\scnitem{\hspace{-0.7cm}Abstract sc-agent for finding direct links between
two objects}
\scnitem{\hspace{-0.7cm}Abstract sc-agent for searching concepts through
which a given concept is defined}
\scnitem{\hspace{-0.7cm}Abstract sc-agent for searching the scope of a
relation definition}
\scnitem{\hspace{-0.7cm}Abstract sc-agent to find a definition or
explanation for a given object}
\scnitem{\hspace{-0.7cm}Abstract sc-agent for finding examples for a
given concept}
\scnitem {\hspace{-0.7cm}Abstract sc-agent for finding a formal statement
record for a given statement sign}
\scnitem{\hspace{-0.7cm}Abstract sc-agent for finding illustrations for a
given object}
\scnitem{\hspace{-0.7cm}Abstract sc-agent of finding key sc-elements for
a given subject area}
\scnitem{\hspace{-0.7cm}An abstract sc-agent searches for concepts that
are defined on the basis of a given}
\scnitem{\hspace{-0.7cm}Abstract sc-agent search for all constructs
isomorphic to a given pattern}
\scnitem{\hspace{-0.7cm}Abstract sc-agent for finding the sc-text of a
proof for a given assertion}
\scnitem{\hspace{-0.7cm}Abstract sc-agent for searching relations defined
on a concept}
\scnitem{\hspace{-0.7cm}Abstract sc-agent for searching sc-text of
condition and problem solution}
\scnitem{\hspace{-0.7cm}Abstract sc-agent for searching statements about
an object}
\end{scnrelfromset}
\par The following is a decomposition of the Abstract nonatomic sc-agent problem solver.
\vspace{0.45cm}
\\  \textbf{\normalsize{Abstract non-atomic sc-agent problem solving agent}}
\begin{scnrelfromset}{\hspace{-0.7cm} decomposition of an abstract sc-agent}
\scnitem {\hspace{-0.65cm}Abstract sc-agent for searching the value of an
unknown quantity}
\scnitem{\hspace{-0.8cm} Abstract sc-agent for verifying the truth of an
assertion}
\scnitem{\hspace{-0.8cm} Abstract sc-agent application of problem-solving
strategies}
\scnitem{\hspace{-0.8cm} Abstract sc-agent of performing logical inference}
\scnitem{\hspace{-0.8cm} Abstract non-atomic sc-agent for calculating
mathematical expressions}
\begin{scnrelfromset}{\hspace{-0.5cm}decomposition of an abstract sc-agent}
\scnitem{\hspace{-0.5cm}Abstract sc-agent for coordinating the
calculation of mathematical expressions}
      \scnitem{\hspace{-0.65cm}Abstract sc-agent for degree expansion,
root extraction and finding the natural logarithm}
      \scnitem{\hspace{-0.65cm}Abstract sc-agent for addition and
subtraction of quantities and numbers}
\scnitem{\hspace{-0.65cm} Abstract sc-agent of product and division
of quantities and numbers }
\scnitem{\hspace{-0.65cm} Abstract sc-agent for comparing
quantities and numbers}
\scnitem{\hspace{-0.65cm} Abstract sc-agent for calculating
trigonometric expressions}
\scnitem{\hspace{-0.65cm}Abstract sc-agent of constructing a
strategy for finding a solution to a
problem in width}
      \end{scnrelfromset}
\end{scnrelfromset}
 Abstract sc-agent for coordinating the calculation of mathematical expressions takes a formula as a parameter.
An example is shown below in Fig. 4.
\newpage
\begin{figure}
    \centering
    \includegraphics[width=1\linewidth]{5411207211964818491 (1).jpg}
    \caption{Figure 4. Example formula as input parameter of sc-agent}
\end{figure}
\par \vspace {0.2cm}  A formula is represented as an sc-structure, which contains sc-bindings of mathematical operation relations and sc-nodes, which are signs of numbers or variables whose value is known or to be calculated. In this example,the formula consists of:
\begin{itemize}
 \item variables:
 \begin{itemize} 
\item a,
\item b,
 \item c,
 \item d,
 \item f,
 \end{itemize}
 \item relations:
 \begin{itemize}
\item nrel{\_}sum{\_}of{\_}numbers,
\item nrel{\_}multiplicatin{\_}of{\_}numbers,
\end{itemize}
\item arcs and edges.
\end{itemize}
\par  The abstract sc-agent of coordination of calculation
of mathematical expressions of formula processing after
initiation searches for all sc-edges of relations of arithmetic operations and forms a structure for calling the operation calculation agent with the parameter of the scedges connecting the node of the relation and the binary
arc of the basic kind. In turn, each of the sc-agents for
operation computation checks whether it can compute
an operation of the given type. If it can, it computes
the operation and creates an sc-node with the answer,
otherwise it does not continue its work.
\par  Thus, the abstract sc-agent coordinating the computation of mathematical expressions processing formula does not know in advance which agent to call specifically.All agents react to the initiated action by checking the input parameters as the initial condition of the problem.
\par  The values of the variables in the formula can be specified in advance, or they can be found in the course of solving the problem. Below are the steps of calculating the values of the variables in the formula, if the values of a, d, c are known in advance (otherwise the formula would not have a specific value).
 \par Since the values of a and b must be known to compute f, the sc-agent checks if their values are known. Since the value is known only for a, the agent will generate a
 \vspace{1.5cm}
 \begin{figure}
    \centering
    \includegraphics[width=1\linewidth]{5411207211964818541.jpg}
    \caption{Figure 5. Example formula as input parameter for sc-agent}
\end{figure}
\begin{figure}[h]
    \centering
    \includegraphics[width=1\linewidth]{5411207211964818542.jpg}
    \caption{Figure 6. Example of operation calculation agent initiation}
    \label{fig:table}
\end{figure}
structure to initiate the agent to compute an arithmetic operation, after which the value of b will be known.
\par  An example of such a structure for initiating the scagent is given below in fig \ref{fig:table}
\\  Once completed, the agent will create the following
construct in the KB. 7
\par  Thus, if all initiated agents are successfully executed,the value of the value of f.
\par  This agent can be used to calculate the values of area,perimeter, etc. using predetermined formulas
\begin{figure}[h]
    \centering
    \includegraphics[width=0.9\linewidth]{5411207211964818543.jpg}
    \caption{Figure 7. Example of the result of the operation calculation agent execution}
\end{figure}
\newpage
\hspace{2.5cm} \RomanNumeralCaps{4}. Conclusion
\par The paper proposes an approach to the development
of methods and means of constructing plans for problem
solving in ISs, which allows us to improve the accuracy
of answers, as well as to overcome the shortcomings of
modern ISs.
\par The proposed model allows us to consider the developed problem solver at different levels of detail, which
provides the possibility of step-by-step design of solvers,
as well as their modifiability.
\par Classification and specification of actions, problems
are specified.
\par The architecture is considered and the IS itself, realizing the proposed approach, is described.
\par The obtained results will allow to increase the efficiency of designing ISs and means of automating the
development of such systems, as well as to provide an
opportunity not only for the developer, but also for the
IS to automatically supplement the system with new
knowledge and skills.
\vspace{0.2cm}
\par \hspace{2.5cm} Acknowledgment
\vspace{0.2cm}
\par The authors would like to thnk the research groups of
the Department of Intelligent Information Technologies
of the Belarusian State University of Informatics and Radioelectronics
\vspace{-0.5cm}
\begin{thebibliography}{196}
\setlength{\itemsep}{0pt}
\setlength{\parskip}{0pt}
\footnotesize{
\bibitem  I. Z. Batyrshin, Fuzzy hybrid systems. Theory and practice,
H. G. Y. M., Ed. Physmatlit, 2007.
\bibitem T. Pratt, Programming languages: development and realization
: transl. from English / T. Pratt, M. Zelkovits, A. Matrosov, Ed.
SPb. Peter Print, 2002.
\bibitem L. A. Gladkov, Genetic algorithms : textbook, V. M. K. n. r.
L. A. Gladkov, V. V. Kureichik and supplementary ed. M., Eds.
Physmatlit, 2006.
\bibitem V. V. Emelyanov, Theory and practice of evolutionary modeling,
V. M. K. V. V. Emelyanov, V. V. Kureichik, Ed. Physmatlit,
2003.
\bibitem M. B. Berkinblit, Neural networks : an experimental textbook,
M. B. Berkinblit, Ed. MIROS : All-Russian extramural multidisciplinary school of the Russian Academy of Sciences., 1993.
\bibitem V. A. Golovko, Neural networks: training, organization, and
applications : a textbook, V. A. Golovko, Ed. Journal. "Radiotekhnika", 2001.
\bibitem A. N. Gorban, Neural networks on a personal computer, D. A. R.
A. N. Gorban, Ed. Novosibirsk : Nauka, 1996.
\bibitem V. N. Vagin, Reliable and plausible inference in intelligent
systems, 2nd ed., D. A. P. B. N. Vagina, Ed. Fizmatlit, 2008.
\bibitem B. A. Kulik, The logic of natural reasoning, B. A. Kulik, Ed. St.
Petersburg. : Nev. dialect, 2001.
\bibitem D. Poya, Programming languages: design and implementation :
transl. from English, M. Z. e. b. A. M. T. Pratt, Ed. Peter Print,
2002.
\bibitem I. Z. Batyrshin, Basic operations of fuzzy logic and their
generalizations, I. Z. Batyrshin., Ed. Kazan : Fatherland, 2001.
\bibitem N. P. Demenkov, Fuzzy control in technical systems : textbook,
N. P. Demenkov, Ed. Moscow State Technical University
Publishing House, 2005.
\bibitem D. A. Pospelov, Modeling reasoning: experience in analyzing
thought acts. Radio and communications, 1989.
\bibitem R. A. Reiter, A logic for default reasoning, R. Reiter, Ed. Artificial Intelligence., 1980, vol. 13, no. 13.
\bibitem  A. P. Eremeev, Construction of ternary logic-based decision
functions in decision-making systems under uncertainty, A. P.
Eremeev, Ed. Famous Russian Academy of Sciences. Theory
and systems of management., 1997.
\bibitem  A. Tamkin, M. Brundage, J. Clark, and D. Ganguli,
“Understanding the capabilities, limitations, and societal impact
of large language models,” CoRR, vol. abs/2102.02503, 2021.
[Online]. Available: https://arxiv.org/abs/2102.02503
\bibitem  K. Bantsevich, M. Kovalev, and N. Malinovskaya, “Integration
of large language models with knowledge bases of intelligent
systems,” Otkrytye semanticheskie tekhnologii proektirovaniya intellektual’nykh system [Open semantic technologies for intelligent
systems], pp. pp 213–219, 2023.
\bibitem V. V. Golenkov and N. A. Gulyakina, “Next-generation intelligent
computer systems and technology of complex support of their
life cycle,” Otkrytye semanticheskie tekhnologii proektirovaniya
intellektual’nykh system [Open semantic technologies for intelligent systems], pp. 27–40, 2022.
\bibitem  M. V. Kovalev, “Convergence and integration of artificial neural
networks with knowledge bases in next-generation intelligent
computer systems,” Otkrytye semanticheskie tekhnologii proektirovaniya intellektual’nykh system [Open semantic technologies
for intelligent systems], pp. 173–186, 2022.
\bibitem (2023, March) Opinion | Noam Chomsky: The False Promise of
ChatGPT. [Online]. Available: https://www.nytimes.com/2023/
03/08/opinion/noam-chomsky-chatgpt-ai.html
\bibitem  V. Golenkov, N. Gulyakina, and D. Shunkevich, Open technology
for ontological design, production and operation of semantically
compatible hybrid intelligent computer systems, G. V.V., Ed.
Minsk: Bestprint, 2023.
\bibitem  A. N. Averkin, A Comprehensive Dictionary of Artificial
Intelligence, H. G. Yarushina, Ed. Radio and communications,
1992.
\bibitem “Github:scl-machine [electronic resource].” [Online]. Available:
https://github.com/ostis-ai/scl-machine
}
\end{thebibliography}
\begin{center}
\subsection*{МЕТОДЫ И СРЕДСТВА ПОСТРОЕНИЯ
ПЛАНОВ РЕШЕНИЯ ЗАДАЧ В
ИНТЕЛЛЕКТУАЛЬНЫХ СИСТЕМАХ
НА ПРИМЕРЕ ИНТЕЛЛЕКТУАЛЬНОЙ
СИСТЕМЫ ПО ГЕОМЕТРИИ}
\end{center}
\Large{Малиновская Н. В., Макаренко А. И.}
\vspace{0.4cm}
\normalsize{
\par В данной работе предлагается подход к разработке
методов и средств построения планов решения задач
в интеллектуальных системах на примере интеллектуальной системы по геометрии. Описанный подход
направлен на повышение точности ответов засчет
возможности декомпозиции задач на более простые,
а также направлен на преодоление недостатков современных интеллектуальных систем. Описана интеллектуальная система, реализующая предлагаемый подход.
\vspace{1.5cm} \hspace{5cm} Received 01.04.2024
}

\end{document}
