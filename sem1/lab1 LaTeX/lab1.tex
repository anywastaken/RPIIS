\documentclass[10pt, a4paper]{article}

\usepackage[utf8]{inputenc}
\usepackage[russian]{babel}
\usepackage{multicol}
\usepackage{enumitem}

\usepackage[left=1.9cm,right=1.9cm, top=2.2cm,bottom=2.5cm]{geometry}

\setlength{\columnsep}{0.5cm}
\setcounter{page}{123}

\begin{document}
    \begin{multicols}{2}
        \noindent
        or more information systems or components to exchange information and to use information obtained as a result of the exchange”. With the addition of the ability to discuss and negotiate, this concept can be extended to both users and development engineers who work on creating information systems for various purposes, involving in the development process specialists from areas related to information technology, as well as from those areas, for which this or that system is being developed. And in their work, it is the ability to interact, i. e., interoperability, that plays a key role in obtaining a high-quality new product.
        
        Currently, not a single area of human activity can do without the use of information and communication technologies. The interoperability is one of the basic properties, without which further formation and development of the information society will become impossible. Today, in all spheres of human activity, there is an ever-accelerating transition to working with a large number of information systems, network resources, and computing power. This leads to interpenetration and the necessary increasingly conscious and deep interaction of knowledge of specialists from various fields, to the need to understand requests and solutions, and collaborate on tasks. And the level of such interaction will only increase with increasing diversity and integration of knowledge.
        \newline
        
        \noindent
        \textit{A. The role of intelligent information technologies in the development of cognitive abilities, emotional intelligence and interoperability}
        \newline
        
        Let us dwell further on the impact of the use of digital information space on learning. Today, educational systems are built on the basis of methods and technologies of artificial intelligence, which is increasingly penetrating our lives, including education. All of the above puts forward additional requirements for the information systems used in the learning process.

        Artificial intelligence, of course, gives us access to broad and detailed information for a various range of issues, but most information systems are organized as search engines. The introduction artificial intelligence technologies, including neural network technologies, into the educational process and human learning activities is a very successful and promising solution. Far instance, there are more than 200 publicly available neural networks. Certain types of neural networks are designed to search for information, check and edit texts, generate images, create curricula, programs and presentations. These neural networks are capable to help teachers draw up curriculum, create tests, organize the structure of a lesson/lecture/seminar, and provide assistance in assessing knowledge. For students, they can help with identifying topics to study and selecting materials on these topics [17].
        
        A very illustrative example of how artificial intelligence should be used in training is the example given by Eric Ofgang [18] when comparing the Gemini and
        ChatGPT neural networks. The author of the article asked the neural network to write an essay for him on a certain topic. In response, the Gemini chatbot suggested a possible outline for the essay. When asked again to write the text, it offered to help as a “writing coach” and gave some tips on how to write an essay on a given topic himself. This example clearly demonstrates progress in the development of approaches to the use of intelligent technologies for obtaining and processing information in the learning process. Neural networks should not do the original, creative part of the work for a person. They should be assistants both in searching for information and in learning, acquiring new knowledge and skills. The same approach should be implemented in general when creating intelligent teaching systems for any level of education.
        
        \begin{center}
             \MakeUppercase{\romannumeral3}. Semantic and cognitive-emotional approach as a basis for educational intelligent systems building
        \end{center}
        
        In our days there is no doubt about the crucial necessity of the intelligent learning systems application at all levels of the education system. Along with that the wide variety of different technologies of such systems construction and operation drives to a number of problems for their actual application at schools — the users are demanded to start from the beginning each time with some new system using some new technology. Thus, the problem has its solution - usage of common technology to systems development and application, which would give an opportunity for the users with different skill-level apply such systems while their studies. Thus, the interoperability is not only technical systems property. It should be considered as one of the main skills of any specialist, and the basis for such skills must be formed within the school education process. Considering the issue of developing emotional intelligence and interoperability along with cognitive abilities, we can consider options for using modern intellectual technologies in this direction at school. The first option is to develop a single digital platform that would integrate various educational resources, applications and tools, providing students and teachers with the ability to access a variety of materials and tasks without the need to constantly switch between different systems.
        
        The OSTIS-Technology [19] is the example of such a basic, common technology of interoperable systems building. It provides a unified semantic platform making it possible to exchange results and data, not only within the technical part of the educational systems, but also present information to the users — children, teachers and parents — and ensure interoperability among them, as long as among the specialists and the public.
        
        Within the framework of this work, it is proposed to take as a basis the approaches to ontology development. Some main features of the cognitive skills corresponding and deeply influencing the development of emotional intelligence and interoperability of the school students will be discussed as the objects for formal clarifying and agreement within the framework of the corresponding set of semantic interpretation.

        \begin{center}
            \MakeUppercase{\romannumeral4}. The process of cognition, the development of emotional intelligence and interoperability
        \end{center}
        
        To formalize the basics of any field it is important to find those that play the most important role and agree on their understanding. Considering the interoperability as a property of a student we should clarify the essential aspects that impact its formation. In [20], we have already presented some views on this point — the individual approach to learning, planning, drawing up programs at various levels of education and interdisciplinary connections. But without the ability of students to perceive this information, there will be no result; at graduation we will not see a trained person capable of solving the problems facing him. Therefore, now we would like to touch on some general approaches and methodological techniques aimed at more successful acquisition of knowledge and development of mental executive abilities of school students. The main points that we would like to draw attention to are the modeling of reality, associativity, completeness and consistency of knowledge, stimulation of cognition.

        Let’s first consider reality modeling. When the new material taught or the formulas used are related to what is encountered and can be applied in real life, then such material is learned much more successfully. The child learns to count, divide, etc. on apples, candies, just common things he uses every day. This skill remains firmly in his head for the rest of his life. Therefore, at school, even in high school, it is necessary to provide material that is confirmed directly or indirectly by examples from life, which the student can discuss with parents and friends. There is no need to strive to present the latest scientific achievements at school, even if they are super important for some area of knowledge. For example, if you present in a botany lesson about 20 layers that make up the bark of a tree, then a diligent student will, of course, memorize their names to answer the test, but they will not remain in memory for a long time, because there is no confirmation in ordinary life around them, there is no immediate interest in such depth of the issue. Unless this student is deeply interested in botany, but this is a different case. Some researchers attribute better performance in humanitarian subjects and the fact that the majority of graduates want to connect their future activities with the humanitarian field precisely to the fact that humanitarian subjects at school, unlike natural sciences, consider issues that are close to most people and related life situations.

        Another approach to teaching methods is directly related to the modeling of reality — associativity. If you learn to remember new information based on associations, then this process will become faster, more effective for storing in long-term memory and applying this knowledge at the right time.
        
        The issues considered in the learning process must be complete and complete at the level of possible perception of the audience. You cannot give part of the conclusion, then reduce it to the fact that the formula is too complex and therefore we use a simple approximation or do not take this into account.

        Typically, educational material is taught in sequence: a descriptive part, applied formulas and laws, and completed tasks. There must also be feedback. This means that when going through subsequent topics, we must return to this material in order to justify new information on its basis. Repeated repetition of material over an increasing period of time, application of previously studied material in other topics, subjects and complex tasks is one of the foundations for its successful memorization and assimilation.

        Also, for a greater degree of involvement in the learning process in order to obtain a higher level of knowledge, it is necessary to use various types of stimulation for cognition — additional points for completing a task, access to the next tasks, the next level (additional functions, like in games), competition tables for completing and leaderboards, etc.

        Interoperability and emotional intelligence are two key aspects that are essential in school education today. Emotional intelligence refers to the ability to manage your own emotions and the emotions of others. In school education, it becomes important because it helps students develop skills in self-regulation, empathy, social competence and conflict resolution. Emotional intelligence training can be built into curricula through social-emotional learning programs that include skills such as managing stress, developing an understanding of emotions, and how to interact effectively with others.

        Integrating interoperability and emotional intelligence in school education can lead to more flexible and adaptive educational environments that take into account the needs and individual characteristics of each student. Technology can assist in this process by providing access to a variety of educational resources and tools, as well as supporting the development of social-emotional skills.

        The use of common data exchange standards to transfer information about the progress, learning process and individual needs of students between different educational platforms will allow, for example, data on a student’s progress in any online course to be automatically transferred to the school information system for teachers. The use of adaptive learning algorithms in intelligent teaching systems will make it possible to analyze data on a student’s performance and learning style, and then offer personalized tasks and recommendations. Such systems can be integrated with various educational platforms to ensure a continuous flow of data. At the same time, the use of interoperable tools for students to collaborate on projects and exchange materials, such as documents, presentations and videos in real time, will contribute to the development of interoperability among students in the process of interacting with each other and with the system itself.

        In addition to directly teaching functions, intelligent systems in educational activities can also perform administrative and technical ones. For example, implement adaptive management of classes and schedules. The use of a classroom management system, which can be integrated with a class schedule, electronic journal and other educational resources, will ensure effective planning and organization of the educational process in educational institutions of different levels. Thus, interoperability can improve the efficiency and functionality of educational systems, providing more flexible and adaptive learning for students and teachers.

        To develop emotional intelligence, and as a result, interoperability, students can be taught awareness, empathy, social competence, and develop self-regulation and conflict management skills. Mindfulness training will help one develop his ability to consciously perceive his thoughts, emotions, and physical sensations. This may include breathing exercises, exercises for awareness in movement and other techniques that can be taught in physical education classes, creating a culture of interaction with oneself and acceptance of oneself on different levels - physical, emotional, spiritual. This will in turn promote the teaching of empathy and social competence through group exercises, role plays, case discussions or collaborative projects across school disciplines where students are provided with opportunities to understand others’ emotions, develop empathy and improve social interaction skills.

        The development of self-regulation includes learning strategies for managing one’s emotions and reactions in various situations, time management, planning actions in stressful situations, developing problem-solving strategies, etc. Learning conflict management — constructive resolution based on the skills of active listening, expressing one’s emotions and searching compromises. This may include educational games, role-playing games and conflict resolution training.

        Teaching emotional intelligence can be integrated into various academic subjects, for example through analyzing literary characters and their emotional reactions, discussing ethical issues, or studying historical events with an emotional perspective. An important part of this work is also providing support and training to teachers and parents so that they can support the development of students’ emotional intelligence. It should be recognized that for many teachers and parents, the experience of developing emotional intelligence and interoperability is a new, unknown matter that also needs to be learned. To train them, it is possible, for example, to conduct seminars, training, and provide teaching resources to teachers and parents. These methods and strategies can be integrated into the learning process and school life to support the development of students’ emotional intelligence and help them successfully cope with emotional challenges, increasing their competencies, strengthening their abilities, and becoming successful and happy members of society.

        \begin{center}
            \MakeUppercase{\romannumeral5}. Basic concepts to be formalized
        \end{center}

        The task of formalizing concepts in the field of interoperability is complex and requires the participation of psychologists, physiologists and teachers. The list of the items to formalize at first stages can be formed according to the discussion above. Two ways are possible - from the definition of the emotional intelligence and cognitive properties down to the basics, and the other one - from the main conceptual terms up to the under- standing of emotional concepts and interoperability. Thus the definitions of the concepts like perception, learning, memory, understanding, awareness, reasoning, judgment, intuition should be determined and agreed. The next step is to derive the contribution of each term to the whole understanding of the level of the cognitive skills, as well as the methods to determine this level. There will appear a hierarchical structure of concepts leading to a description of the phenomenon (ability) of cognition.

        The human side of interoperability is a hard part to define. It requires time, energy, and money [5]. The words used here are not basics. There is a need to make a precise research on the issues that contribute to each of them and formalize their meanings, make them acceptable for the information system determination and measuring.

        Considering the emotional intelligence we are talking about such concepts as self-awareness, self-regulation, motivation, empathy, and social skills, which are complex and require some investigations on a possibility to identify both their formal meanings as well as options and principals of their physical measuring the degree of their manifestation in a person.
        
        \begin{center}
            \MakeUppercase{\romannumeral6}. Заключение
        \end{center}

        The ability for interoperable behavior is not genetically embedded in people. It must be raised, starting from an early age, in accordance with children’s ability to perceive tasks that require interaction and joint activity to solve them. Interoperable behavior requires mastery of conceptual apparatus from different scientific fields, the ability to formulate descriptions of phenomena, problems, tasks that a person faces, ways to solve them, taking into account the interests of different people, the characteristics of their thinking and behavior. education should not just provide knowledge in individual sciences, but educate a cognitive personality
    \end{multicols}
\end{document}