\documentclass[a4paper]{article}

\usepackage{multicol} %делаю колонки
\usepackage{setspace} %межстрочный интервал
\usepackage{fancyhdr} %настройки верхнего и нижнего колонтитулов в документе.
\usepackage{newtxtext, newtxmath} % шрифт
\usepackage{hyphenat} %переносы
\usepackage{scrextend} % настройка полей, выравнивания и т.д. размер шрифта
\usepackage{enumitem} % настройка списков
%\usepackage{mathtools} % мат выражения
\usepackage[left=2.5cm,right=2.5cm,top=2.5cm,bottom=2.5cm]{geometry}
\setlist[itemize]{noitemsep, topsep=0pt} % отступы в itemize
\fancyhf{} % очищает все верхние и нижние колонтитулы.
\renewcommand{\headrulewidth}{0pt}
\cfoot{\textbf{\thepage}} % жирные номера страниц
\pagestyle{fancy}
\setcounter{page}{129} % настройка нумерации страниц
\setlength{\columnsep}{.5cm} % интервал между колонками

\begin{document}

\setlength\parindent{11pt}
\fontsize{9.7}{13}\selectfont

\begin{multicols}{2}

{\setlength{\parindent}{0pt} % Убрать в самом начале отступ
 \nohyphens{
goal through mutual benefit. People co-operated and coordinated their actions on the basis of common interests,
traditions and customs, which helped to coordinate actions and establish rules of interaction between people.
The first languages appeared, in which people could express their thoughts and transmit these thoughts to other
people. People began to organise themselves into groups,
thanks to which they became stronger among other
groups of people and nature in general. In each group, socalled leaders were formed, usually distinguished among
other people by their physical and mental abilities. Thus
a form of people management appeared — power, with
the help of which one person could organise the work
of other people. This form of people management has
evolved a lot since then.}}

{\fontsize{9.5}{13}\selectfont Today, people manage not just people, but the environment in which they exist: resources, relationships,
knowledge, etc. Whereas in ancient times the question of
survival was of primary importance to humans, today the
main question is how to properly accumulate, organise,
use and transfer knowledge from one person to another
and from one generation to another generation of people.
This question is also related to the question of human
survival. If we, people, do not or cannot develop, i.e.
accumulate and multiply the knowledge that we have,
how will we be able to solve those constantly emerging
problems that any person or society faces, how will we
be able to build a future in which every person will be
happy and will be able to get what he or she wants. Even
today, questions about the future of humanity remain a
priority and open for discussion.
\par}

\vspace{1.0mm}

{\setlength{\parindent}{0pt}\nohyphens{\textit{B. Organisation and consistency of collective activities through standardisation and legal regulations }}} \par

\vspace{0.5mm}

{\fontsize{9.5}{13}\selectfont For a long time of human and social existence, people
have invented and realised quite a large number of forms
by means of which they can communicate with each
other, achieve goals by reaching common understanding
and agreement, solve problems, relying on common
sources of knowledge that have passed through time. The
issue of organising common activities between people remains the most important. It is the coordinated collective
activity of people that contributes to the rapid flow of the
creative process, which allows to solve problems faster.\par}

{\fontsize{9.6}{13}\selectfont Compared to primitive society, the level of organisation is much higher nowadays. Nowadays, people use
a great variety of methods and tools to organise and
coordinate their activities:
\par}

\begin{itemize}[leftmargin=5mm]
        
        \item organisational structures are established within businesses, organisations and public institutions that define the hierarchy, roles and functions of employees;
        \item information technologies such as e-mail, messengers, video conferencing, project management systems and others are used to exchange information,
coordinate actions between people;
        \item all rules of behaviour and processes within communities are standardised to improve efficiency and
ensure the quality of their products or services.
    
\end{itemize}

{\fontsize{9.5}{13}\selectfont It can be seen that the degree of the consistency of
actions is also much higher compared to earlier stages of
human development.}

{\fontsize{9.5}{13}\selectfont Regulations and standards were the first to address the
problem of collecting and systematising knowledge, as
they represent formalised norms and rules that regulate
the behaviour of people and organisations in society. The
introduction of regulations and standards helps to create a
unified and ordered approach to knowledge organisation
and provides a common basis for information exchange.}

{\fontsize{9.5}{13}\selectfont Regulations and standards enable:}

\begin{itemize}[leftmargin=5mm]
        
        \item to ensure unity, accuracy and reliability of information;

        \item to organise knowledge, i.e. classify and structure
information;

        \item to protect people’s rights and interests.
    
\end{itemize}

{\fontsize{9.5}{13}\selectfont Regulations and standards are documents that establish rules, norms and requirements in a particular area
(e.g. product quality standards, building codes, safety
regulations, etc.). They are aimed at ensuring uniformity,
quality and safety in different areas of activity. The task
of any standard in general is to describe a consistent
system of concepts (and corresponding terms), business
processes, rules and other regularities, ways of solving
certain classes of tasks etc.
}

{\fontsize{9.5}{13}\selectfont However, they cannot completely solve all existing
problems because:
}

\begin{itemize}[leftmargin=5mm]
        
        \item Knowledge is constantly evolving and changing,
making regulations and standards quickly outdated
or unable to adequately reflect new knowledge.
        \item The process of adding new knowledge is too
resource-intensive because adding new knowledge
requires searching for similar existing knowledge
and manually integrating that knowledge with existing knowledge.
        \item Modern normative legal acts and standards describe
only rather narrowly specialised knowledge, when
the rest of knowledge is not standardised in any way.
That is, the question of interdisciplinary organisation of knowledge remains open.
        \item The most urgent problem remains the problem related not to the form, but to the essence (semantics)
of standards - the problem of inconsistency of systems of concepts and terms between different standards, which is relevant even for standards within
the same field of activity.
    
\end{itemize}

\vspace{0.7mm}

{\setlength{\parindent}{0pt}\nohyphens{\textit{C. Use of common encyclopaedias in organising and
coordinating collective activities
}}} \par

\vspace{0.1mm}

{\fontsize{9.5}{13}\selectfont In addition to regulations and standards, there are
encyclopaedias [9] and dictionaries that cover knowledge
from different subject domains and are interdisciplinary
}

\newpage \nohyphens{\fontsize{9.5}{13}\selectfont\setlength{\parindent}{0pt}[10]. Encyclopaedias and dictionaries are designed to
integrate information related to a common topic.} \par

{\fontsize{9.5}{13}\selectfont \nohyphens{Encyclopaedias are reference publications that contain
information on a wide range of topics and subjects. They
are intended for general familiarisation with different
fields of knowledge and may contain general information,
historical facts, descriptions of phenomena, etc. The
purpose of any encyclopaedia is to collect knowledge
scattered across disciplines and bring it into a system
understandable to the individual. \par}}

{\fontsize{9.5}{13}\selectfont \nohyphens{While regulations and standards cannot always provide
complete information or cover all aspects of a particular
topic, encyclopaedias can be useful to provide a broader
context or additional information on a given topic. Today,
regulations, standards and encyclopaedias complement
each other, but are also used for different purposes.\par}}

\vspace{1.5mm}

{\setlength{\parindent}{0pt}\nohyphens{\textit{D. Digitalisation of information for simplifying and accelerating the organisation and consistency of collective
action
}}} \par

\vspace{0.9mm}

{\fontsize{9.5}{13}\selectfont \nohyphens{With the development of technology, it has become
much easier to describe and popularise knowledge by
creating topic-based websites. One of the representatives
of this kind of knowledge repository is mathprofi — a
site that is a resource describing all topics of school and
higher mathematics. There is a large number of such
online resources on any topic and subject. In them, information is presented and described in an understandable
and accessible form for any untrained reader. \par}}

{\fontsize{9.5}{13}\selectfont \nohyphens{In parallel with the development of sites like mathprofi,
there occurred an idea of creating a common repository
that brings together all the knowledge of mankind, categorised by topic, and to which any person can add new
information. \par}}

{\fontsize{9.5}{13}\selectfont \nohyphens{The most widely used computer encyclopaedia at
present is Wikipedia — a publicly available multilingual
universal Internet encyclopaedia with free content. Its
main advantages are its multilingualism and the possibility for users to add and adjust its content. \par}}


{\fontsize{9.5}{13}\selectfont \nohyphens{Traditional wikis based on this approach have a number of disadvantages, including a lack of content consistency, that is, the lack of uniformity in the presentation and formalisation of this content. In wikis, due
to frequent duplication of data, the same information
may be contained on several different pages. When this
information is changed on one wiki page, users must
ensure that the data is also updated on all other pages. \par}}

{\fontsize{9.5}{12.6}\selectfont \nohyphens{Another disadvantage is the difficulty of accessing
the knowledge available on wikis. Large wikis contain
thousands of pages. Performing complex search queries
and comparing information from different pages in traditional wiki systems is a time-consuming task. Traditional
wikis use flat classification systems (tags), or classifiers
organised into taxonomies. The inability to use typified
properties generates a huge number of tags or categories
[9].
 \par}}

{\fontsize{9.6}{13}\selectfont \nohyphens{Open databases and knowledge bases are promising
tools for building and retrieving knowledge. Semantic
Web-based open databases and knowledge bases [11] are
resources that use Semantic Web principles and technologies to represent and organise data and knowledge
in a structured and semantic form [12]. The Semantic
Web is an extension of the World Wide Web in which
information has semantic meaning that allows computers
to understand and process data efficiently and accurately.
Examples of such Semantic Web-based databases and
knowledge bases are: \par}}

\begin{itemize}[leftmargin=5mm]
        
        \item DBpedia — one of the best known Semantic Web
projects. It extracts structured data from Wikipedia
and presents it in RDF format. DBpedia contains
an extensive set of knowledge, including information about people, places, organisations, scientific
articles and more.
        \item Linked Open Data (LOD) is an initiative that aggregates and provides access to open data in RDF
format from a variety of sources. LOD brings together data from fields such as geography, biology,
culture, economics and others, and makes it possible
to share and analyse these data
        \item Wikidata is an open knowledge base developed by
the Wikimedia Foundation. It contains structured
data about various entities, including people, places,
books, films, scientific terms, and more. Wikidata
uses the RDF language to represent the data and
provides an API to access this data.
        \item Cyc — a project to create an ontological knowledge
base that allows computer systems to solve complex
Artificial Intelligence problems based on logical
inference.
        \item GeoNames — a geographic database that contains
information about places from all over the world.
It provides data on geographic coordinates, population, administrative units, geographic objects, and
other information. GeoNames uses Semantic Web
standards such 
        \item MusicBrainz — an open source music database that
contains information about music artists, albums,
tracks and other music-related entities. It uses Semantic Web technologies to organise and represent
data, and provides an API to access this data.
    
\end{itemize}

{\fontsize{9.6}{13}\selectfont \nohyphens{The listed databases and knowledge bases also do not
solve all the problems that modern Wikipedia has. \par}}

{\fontsize{9.6}{13}\selectfont \nohyphens{Unfortunately, modern traditions of presentation of various kinds of documentation, standards, reports, scientific
and technical articles and monographs are not only not
oriented to their adequate understanding by intelligent
computer systems, but also do not contribute to their
quick understanding by those people to whom these
texts are addressed. The latter circumstance requires the
development (writing) of textbooks and teaching aids \par}}

\newpage \nohyphens{\fontsize{9.7}{13}\selectfont\setlength{\parindent}{0pt} specially designed for those people who are beginning
to master the relevant field of knowledge, who have
not yet acquired the necessary qualifications. But it is
obvious that this implies a significant duplication of the
information presented.} \par

\vspace{2.9mm}

{\setlength{\parindent}{0pt}\nohyphens{\textit{E. Shortcomings of current solutions to ensure consistency and compatibility of different activities
}}} \par

\vspace{1.5mm}

{\fontsize{9.7}{13}\selectfont \nohyphens{All the tools considered for representing, structuring and accumulating information make it possible to
simplify the organisation and coordination of collective
activities, but do not allow to solve these tasks in a
comprehensive way, because: \par}}

\begin{itemize}[leftmargin=5.5mm]
        
        \item The increase in the number of reference materials
presenting and describing the same information in
different forms leads to an increase in duplication
and, consequently, inconsistency of this information.
        \item There is quite a lot of information in existing information resources that is characterised by inaccuracy,
unstructured, incomplete, incoherent and unreliable
information.
        \item Information becomes obsolete rather quickly, i.e. becomes irrelevant and unclaimed due to finding new
methods of solving existing problems. All irrelevant
information is quickly accumulated in the Internet
space. That is why there is a lot of so-called "junk
information" in Internet resources, which is this
irrelevant and unclaimed information.
        \item The input of information in information resources is
done by intermediaries - people who do not have the
necessary competence to modernise and disseminate
this information, which directly affects the quality of
all information. This is also due to the fact that a
person who receives information from one source
interprets and transmits it to another source in his
or her own way. Different people describe concepts
from different sources by synonymous terms, which
leads to the loss of the original meaning of these
concepts. Thus, new contradictions in information
appear.
        \item Working with huge amounts of information implies
working with several sources of this information. In
such sources it is difficult to search for necessary
(relevant) information, as there is a huge number
of different categories, which implies the use of
complex search operations.
        \item This, in turn, is related to the language of knowledge representation. The format of knowledge representation and description in reference materials is
understandable only to a human being and cannot
be processed by a computer system, and as a consequence, cannot be used for solving problems by a
computer.
        \item Knowledge is most often structured in the form of
books, encyclopaedias, dictionaries and reference
books on specific subject domains, which allows
one to learn a particular subject domain quickly.
However, this makes it difficult to understand information at the "junctions" of subject domains,
so that a person is not well-versed in interdisciplinary knowledge. The so-called "mosaicism" of
perception is formed in a human being, as a human
being during training and work gets used to artificial
division of knowledge areas and has difficulties in
solving problems at "junctions".

        \item To integrate information from different sources, algorithms for matching and merging data, identifying
and resolving duplicates, and algorithms for converting to common presentation formats are used, but
even these algorithms do not completely eliminate
inconsistencies and duplication of existing information.

        \item The existing information resources do not standardise and do not apply general principles of presenting
information for a wide range of readers. Each reader
perceives information in his/her own way, and consequently, there are differences in the understanding
of the same information.

        \item The popularisation of knowledge is carried out with
the help of specialised Internet resources that not
only simplify but also distort the presentation of information for professionally untrained readers. The
increase in the number of such Internet resources
contributes to the duplication of information and the
development of contradictions in it.
    
\end{itemize}

{\fontsize{9.5}{13}\selectfont \nohyphens{To solve these problems, methods and technologies
must be utilised which can: \par}}

\begin{itemize}[leftmargin=4.0mm]
        
        \item present any information in the same \underline{same} form;
        \item  \underline{integrate} integrate information from different information
sources;
        \item describe and  \underline{structure} structure information both from one
subject domain and information at "junctions" between subject domains;
        \item  \underline{standardise} standardise the description and visualisation of various types of information;

        \item  \underline{re-use} re-use existing knowledge and accumulate new
knowledge;

        \item present information in a form that is  \underline{understandable}
to both (!) humans and computers;

        \item develop tools to  \underline{quickly find} quickly find the information you
need;
        \item create a  \underline{personalised} personalised experience for any user;
        \item  \underline{develop} methods and tools to improve these methods
and technologies.
        
\end{itemize}

{\fontsize{9.6}{13.0}\selectfont \nohyphens{In other words, it is necessary to create such unified
integrated information resources, with the help of which
it is possible to quickly obtain existing information and
to integrate new information and it would be easy to
coordinate various activities, including activities on the
development of intelligent systems. It is also necessary \par}}

\end{multicols}


\end{document}
