\documentclass[a4paper]{article}
\usepackage{graphicx} % Required for inserting images
\usepackage{fancyhdr}%настройки верхнего и нижнего колонтитулов в документе
\usepackage{setspace} %межстрочный интервал
\usepackage{newtxtext, newtxmath} % Задать шрифт Times New Roman
\usepackage[russian, english]{babel}
\usepackage{hyphenat}
\usepackage{scrextend}
\usepackage{enumitem}
\usepackage{mathtools}
\usepackage{multicol}
\usepackage[left=2.5cm, right=2.5cm, top=2.5cm, bottom=2.5cm]{geometry}
\newcommand{\RomanNumeralCaps}[1]
    {\MakeUppercase{\romannumeral #1}}
    
\setlist[itemize]{noitemsep, topsep=0pt} % убрать отступы itemize
\fancyhf{} % очищает все верхние и нижние колонтитулы.
\renewcommand{\headrulewidth}{0pt} % remove the header rule
\cfoot{\textbf{\thepage}} % жирные номера страниц
\pagestyle{fancy}
\setcounter{page}{147} % настройка нумерации страниц
\setlength{\columnsep}{.5cm} % интервал между колонками
%\title{lab-1}
%\author{Евгений Кулик}
%\date{September 2024}


\begin{document}
\begin{multicols}{2}
\fontsize{10}{13}\selectfont
\begin{itemize}
    \item[-]simplifies the consistency of existing information
and the addition of new information;
    \item[-]standardizes support for methods and tools to
 enhance existing information;
    \item[-] enables the creation of a personalized experience
for any end-user.
\end{itemize}



\par It is also worth noting that the solution proposed
within this paper — \textit{OSTIS Glossary} — is:
\begin{itemize}
    \item a part of the \textit{OSTIS Metasystem} Knowledge base, as
    known the \textit{OSTIS Standard}, which allows:
        \begin{itemize}[leftmargin=3.5mm]
            \item [-]to develop the \textit{OSTIS Glossary} by the same means
by which any intelligent computer system based
on the \textit{OSTIS Technology} is developed;
            \item [-]use the same tools to view and navigate the text
of the \textit{OSTIS Glossary};
            \item [-]automatic consistent development of the \textit{OSTIS
Glossary} and the \textit{OSTIS Standard};
        \end{itemize}
    \item a simplified version of the \textit{OSTIS Standard}, which
allows:
        \begin{itemize}[leftmargin=3.5mm]
            \item[-]to quickly search and reuse existing information;
            \item[-]to quickly provide consistency and integrate new
information;
            \item[-]to reduce the circle of entry for new people to
develop the \textit{OSTIS Technology};
        \end{itemize}
    \item an environment for social and creative learning and
development of new staff in the field of Artificial
Intelligence.\
\end{itemize}
\par The introduction of such information resources can
significantly improve the quality and efficiency of various
activities.
\par The authors believe that this paper will be useful not
only for those who are researching innovative methods
and technologies for more effective organisation of teamwork, but also for those who are just beginning research
in this area.
\begin{center}
{\RomanNumeralCaps{6}. Acknowledgment}
\end{center}
\par The authors would like to thank the scientific staff of
the Department of Intelligent Information Technologies
of the Belarusian State University of Informatics and
Radioelectronics for their help and valuable 
comments.
\begin{center}
\begin{thebibliography}{46}
\fontsize{8}{5}\selectfont
\setlength{\parindent}{0.0cm}
\setlength{\parskip}{0.0cm}
     \bibitem[1]{e1}V. V. Golenkov, N. Guliakina, V. Golovko, V. Krasnoproshin,
“Methodological problems of the current state of works in
the field of artificial intelligence,” in \textit{Otkrytye semanticheskie
tekhnologii proektirovaniya intellektual’nykh system [Open semantic technologies for intelligent systems]}, ser. 5, V. Golenkov,
Ed. BSUIR, Minsk, 2021, pp. 17–24.
     \bibitem[2]{e2}A. M. Ouksel, “Semantic interoperability in global information
systems,” SIGMOD Rec., vol. 28, no. 1, p. 5–12, Mar. 1999.
     \bibitem[3]{e3}V. Golenkov, V. Golovko, N. Guliakina, and V. Krasnoproshin,
“The standardization of intelligent computer systems as a key
challenge of the current stage of development of artificial intelligence technologies,” in \textit{Otkrytye semanticheskie tekhnologii
proektirovaniya intellektual’nykh system [Open semantic technologies for intelligent systems]}, ser. 4, V. Golenkov, Ed. BSUIR,
Minsk, 2020, pp. 73–88.
     \bibitem[4]{e10}V. Golenkov, N. Guliakina, V. Golovko, and V. Krasnoproshin,
“Artificial intelligence standardization is a key challenge for the
technologies of the future,” in \textit{Open Semantic Technologies for
Intelligent System}. Cham: Springer International Publishing,
2020, pp. 1–21.
     \bibitem[5]{e4} P. Hagoort, “Semantic unification,” in \textit{The cognitive neurosciences}, 4th ed. MIT press, 2009, pp. 819–836.

     \bibitem[6]{e5}A. Iliadis, “The tower of babel problem: Making data make sense
with basic formal ontology,” \textit{Online Information Review}, vol. 43,
no. 6, pp. 1021–1045, 2019.
    \bibitem[7]{e6}J. H. Siekmann, “Universal unification,” in \textit{International Conference on Automated Deduction}. Springer, 1984, pp. 1–42.

    \bibitem[8]{e6}A. P. Sokolov and A. O. Golubev, “System of computeraided design of composite materials. part 3. graph-oriented
methodology of development of means of user-system
interaction,” 2021, pp. 43–57.
    \bibitem[9]{e6} G. Evgenev, “Expertopaedia as a means of creating an ontological
internet of knowledge,” \textit{Ontology of Design}`, vol. 9, no. 3 (33), pp.
307–320, 2019.
    \bibitem[10]{e6} V. Polonskiy, “Why do i need a dictionary?” \textit{Nauka i shkola}, no. 1,
pp. 214–226, 2019.
    \bibitem[11]{e6}M. Yahya, J. G. Breslin, and M. I. Ali, “Semantic web and
knowledge graphs for industry 4.0,” \textit{Applied Sciences}, vol. 11,
no. 11, p. 5110, 2021.
    \bibitem[12]{e6}J. Han, S. Sarica, F. Shi, and J. Luo, “Semantic networks for
engineering design: state of the art and future directions,” \textit{Journal
of Mechanical Design}, vol. 144, no. 2, p. 020802, 2022.
    \bibitem[13]{e6} Robinson, I., \textit{Graph databases}. O’Reilly Media, Inc., 2015.
    \bibitem[14]{e6} V. Syskov and V. Borisov, “Method of organisation of collective
activity in complex organisational and technical systems,” \textit{Sovremennye nauchnye issledovaniye i innovatsii}, no. 11, pp. 284–310,
2015.
    \bibitem[15]{e6} W. J. Verhagen, P. Bermell-Garcia, R. E. Van Dijk, and R. Curran,
“A critical review of knowledge-based engineering: An identification of research challenges,” \textit{Advanced Engineering Informatics},
vol. 26, no. 1, pp. 5–15, 2012.

    \bibitem[16]{e6}T. Gruber, “Toward principles for the design of ontologies used
for knowledge sharing,” \textit{Intern. J. of Human-Computer Studies},
vol. 43, no. 5/6, pp. 907–928, 1995.

    \bibitem[17]{e6} A. Palagin and N. Petrenko, “To the question of systemontological integration of subject area knowledge,” \textit{Mathematical
Machines and Systems}, vol. 1, no. 3-4, pp. 63–75, 2007.
    \bibitem[18]{e6}Lim, S.C. Johnson and Liu, Ying and Chen, Yong, “Ontology
in design engineering: status and challenges,” in \textit{International
Conference on Engineering Design 2015 (ICED 2015)}, 07 2015.
    \bibitem[19]{e6}L. Yang, K. Cormican, and M. Yu, “Ontology-based systems
engineering: A state-of-the-art review,” \textit{Computers in Industry},
vol. 111, pp. 148–171, 2019.
    \bibitem[20]{e6}I. Davydenko, V. Tarasov, and A. Fedotova, “Ontological
approach to building knowledge bases based on semantic
networks,” in \textit{Open semantic technologies for designing intelligent
systems (OSTIS-2016) : proceedings of the VI International
scientific-technical conference, Minsk, 18–20 Feb. 2016 y}. Minsk,
2016.
\bibitem[21]{e1}O. Ataeva, N. Kalenov, and V. Serebryakov, “Ontological approach to the description of a unified digital space of scientific
knowledge,” \textit{Electronic Libraries}, vol. 24, no. 1, pp. 3–19, 2021.
     \bibitem[22]{e2}Y. Kravchenko, “Synthesis of heterogeneous knowledge on the
basis of ontologies,” \textit{Investigations of the Southern Federal University. Technical Sciences}, vol. 136, no. 11 (136), pp. 216–221,
2012.
     \bibitem[23]{e3} A. Tuzovskiy, “Development of knowledge management systems
based on a unified ontological knowledge base,” \textit{Investigations of
Tomsk Polytechnic University. Engineering of georesources}, vol.
310, no. 2, pp. 182–185, 2007.
     \bibitem[24]{e10}T. Dillon, E. Chang, and P. Wongthongtham, “Ontology-based
software engineering-software engineering 2.0,” 04 2008, pp. 13–
23.

     \bibitem[25]{e4} V. Ryen, A. Soylu, and D. Roman, “Building semantic knowledge
graphs from (semi-) structured data: a review,” \textit{Future Internet},
vol. 14, no. 5, p. 129, 2022.

     \bibitem[26]{e5}P. Barnaghi, A. Sheth, and C. Henson, “From data to actionable
knowledge: Big data challenges in the web of things [guest editors’
introduction],” IEEE Intelligent Systems, vol. 28, no. 6, pp. 6–11,
2013.
    \bibitem[27]{e6}C. Kellogg, “From data management to knowledge management,”
\textit{Computer}, vol. 19, no. 01, pp. 75–84, 1986.

    \bibitem[28]{e6} G. Fischer and J. Otswald, “Knowledge management: problems,
promises, realities, and challenges,” \textit{IEEE Intelligent systems},
vol. 16, no. 1, pp. 60–72, 2001.
    \bibitem[29]{e6} S. E. Hampton and J. N. Parker, “Collaboration and productivity
in scientific synthesis,” \textit{BioScience}, vol. 61, no. 11, pp. 900–910,
2011.

    \bibitem[30]{e6} I. Sechkina and G. Sechkin, “Synthesis as a goal, method and
    final result of knowledge integration,” \textit{Omskiy nauchny vestnik},
no. 3 (129), pp. 191–192, 2014.
    \bibitem[31]{e6}Y. Zagorulko and O. Borovikova, “An approach to building portals
of scientific knowledge,” \textit{Autometry}, vol. 44, no. 1, pp. 100–110,
2008.
    \bibitem[32]{e6}P. Van Baalen, J. Bloemhof-Ruwaard, and E. Van Heck, “Knowledge sharing in an emerging network of practice:: The role of a
knowledge portal,” \textit{European Management Journal}, vol. 23, no. 3,
pp. 300–314, 2005.
    \bibitem[33]{e6} R. Mack, Y. Ravin, and R. J. Byrd, “Knowledge portals and
the emerging digital knowledge workplace,” \textit{IBM systems journal},
vol. 40, no. 4, pp. 925–955, 2001.
    \bibitem[34]{e6} V. V. Golenkov and N. A. Gulyakina, “Graphodynamic models
of parallel knowledge processing: principles of construction,
implementation and design,” in \textit{Open semantic technologies of
designing intelligent systems (OSTIS-2012) : proceedings of the
II International scientific-technical conference, Minsk, 16–18 Feb.
2012 y}. Minsk, 2012, pp. 23–52.
    \bibitem[35]{e6}V. V. Golenkov, “Ontology-based design of intelligent systems,”
in \textit{Otkrytye semanticheskie tekhnologii proektirovaniya intellektual’nykh system [Open semantic technologies for intelligent
systems]}, ser. Iss. 1. Minsk : BSUIR, 2017, pp. 37–56.

    \bibitem[36]{e6}V. Golenkov, N. Guliakina, I. Davydenko, A. Eremeev, “Methods and tools for ensuring compatibility of computer systems,”
in \textit{Otkrytye semanticheskie tekhnologii proektirovaniya intellektual’nykh system [Open semantic technologies for intelligent
systems]}, ser. 4, V. Golenkov, Ed. BSUIR, Minsk, 2019, pp.
25–52.

    \bibitem[37]{e6}A. Zhmyrko, “Family of external languages of next-generation
computer systems, close to the language of the internal semantic representation of knowledge,” in \textit{Otkrytye semanticheskie
tekhnologii proektirovaniya intellektual’nykh system [Open semantic technologies for intelligent systems]}. Minsk : BSUIR,
2022, pp. 65–80.
    \bibitem[38]{e6}K. Bantsevich, “Metasystem of the ostis technology and
the standard of the ostis technology,” in \textit{Otkrytye semanticheskie tekhnologii proektirovaniya intellektual’nykh system
[Open semantic technologies for intelligent systems]}, ser. Iss. 6,
V. Golenkov, Ed. BSUIR, Minsk, 2022, pp. 357–368.
    \bibitem[39]{e6}V. V. Golenkov and N. A. Gulyakina, “Open project aimed
at creating a technology for component-based design of
intelligent systems,” in \textit{Open Semantic Technologies for Designing
Intelligent Systems (OSTIS-2013) : proceedings of the III
International Scientific and Technical Conference, Minsk, 21-23
Feb. 2013 y}. Minsk, 2013, pp. 55–78.

    \bibitem[40]{e6} A. Zagorskiy, “Principles for implementing the ecosystem of nextgeneration intelligent computer systems,” in \textit{Otkrytye semanticheskie tekhnologii proektirovaniya intellektual’nykh system [Open
semantic technologies for intelligent systems]}. BSUIR, Minsk,
2022, p. 347–356.
  \bibitem[41]{e6} V. Ivashenko, “General-purpose semantic representation language
and semantic space,” in \textit{Otkrytye semanticheskie tekhnologii
proektirovaniya intellektual’nykh system [Open semantic technologies for intelligent systems]}, ser. Iss. 6. Minsk : BSUIR, 2022,
pp. 41–64.

    \bibitem[42]{e6}K. Bantsevich, “Structure of knowledge bases of next-generation
intelligent computer systems: a hierarchical system of subject
domains and their corresponding ontologies,” in \textit{Otkrytye semanticheskie tekhnologii proektirovaniya intellektual’nykh system
[Open semantic technologies for intelligent systems]}, ser. Iss. 6,
V. Golenkov, Ed. BSUIR, Minsk, 2022, pp. 87–98.


    \bibitem[43]{e6} A. Goylo and S. Nikiforov, “Means of formal description of
syntax and denotational semantics of various languages in nextgeneration intelligent computer systems,” in \textit{Otkrytye semanticheskie tekhnologii proektirovaniya intellektual’nykh system [Open
semantic technologies for designing intelligent systems (OSTIS2022)]}. Minsk, 2022, pp. 99–118.
    \bibitem[44]{e6}A. Zagorskiy, “Factors that determine the level of intelligence of
cybernetic systems,” in \textit{Otkrytye semanticheskie tekhnologii proektirovaniya intellektual’nykh system [Open semantic technologies
for intelligent systems]}. BSUIR, Minsk, 2022, p. 13–26.
    \bibitem[45]{e6}O. Parshina, “A question about lexicographic description of
didactic terms,” \textit{Vestnik Volzhskogo universitet n.a. Tatishchev},
no. 7, pp. 34–40, 2011.
    \bibitem[46]{e6} A. Tazetdinov, “Technology of structuring and visualisation of
educational information in tutoring systems,” \textit{Information and
control systems}, no. 1, pp. 60–65, 2009.
\end{thebibliography}
\end{center}
\begin{otherlanguage}{russian}
    \fontsize{10}{15}\selectfont
\begin{center}
\section*{\normalsize{\textbf{\textit{ГЛОССАРИЙ OSTIS} — ИНСТРУМЕНТ
\\ ДЛЯ ОБЕСПЕЧЕНИЯ
\\ СОГЛАСОВАННОЙ И СОВМЕСТИМОЙ
\\ ДЕЯТЕЛЬНОСТИ ПО РАЗРАБОТКЕ
ИНТЕЛЛЕКТУАЛЬНЫХ СИСТЕМ
\\ НОВОГО ПОКОЛЕНИЯ}}}
\fontsize{11}{15}\selectfont
\\
Зотов Н. В., Ходосов Т. П., Остров М. А.,
Позняк А. В., Романчук И. М.,
\\
Рублевская Е. А., Семченко Б. А.,
Сергиевич Д. П., Титов А. В., Шаров Ф. И.
\end{center}
\fontsize{10}{10}\selectfont
\par Данная работа включает подробный анализ проблем
организации различных видов коллективной деятельности,
сравнительный анализ текущих решений по обеспечению
согласованности и совместимости информации из различных
областей знаний, а также анализ методов и технологий
для создания единых информационных пространств для
обеспечения согласованного и совместимого хранения, обработки, накопления и распространения знаний. В работе
предлагается один из вариантов реализации единого информационного ресурса для обеспечения согласованной и
совместимой деятельности по разработке интеллектуальных
компьютерных систем нового поколения— Глоссарий OSTIS.
Описывается его структура, правила структуризации, размещения и идентификации знаний в нём, а также принципы
работы с ним.
\end{otherlanguage}

\begin{flushright} 
Received 15.03.2024
\end{flushright}
\clearpage
\end{multicols}
\begin{center}
\fontsize{22}{30}\selectfont
Fundamentals for the Intelligent\par
Non-Invasive Diagnostics
\end{center}
\begin{multicols}{2}
   \begin{center}
        Natallia Lipnitskaya
\\textit{Department of Intelligent Information Technologies
\\Belarusian State University
\\of Informatics and Radioelectronics}
\\ Minsk, Republic of Belarus
\par Natasha.lipnitskaya@gmail.com
 \end{center}
\begin{center}
  Vladimir Rostovtsev
\par \textit{Republican Scientific and Practical Center}
\par \textit{of Medical Technologies, Informatization,}
\par \textit{Management and Economics of Health Care}
\par Minsk, Republic of Belarus
\par vnrost@kmsd.su
\end{center}
\end{multicols}

\begin{multicols}{2}
\fontsize{10}{13}\selectfont
 \par \textbf{\textit{Abstract}—The article elaborates the needs of the design
and implementation of a intelligent non-invasive diagnostics
system. Technological basis for development and different
variants of non-invasive diagnostics are proposed as two
fundamental components of such system.}
\par \textbf{The domestic Open Semantic Technology of Intelligent
Systems (OSTIS) is proposed to be used as a core technological foundation while designing the intelligent diagnostic
system. The adaptation of diagnostic tasks within logicalsemantic approach will allow to carry out differential
diagnostics (i. e. formulating several diagnostic hypotheses).
Various approaches towards the non-invasive diagnostics
have been considered: functional-spectral diagnostics (FSDdiagnostics), bioimpedance analysis, preliminary diagnostics based on the assessment of the basic parameters
of functional state, diagnostics by Zakharyin-Ged zones,
diagnostics by Nakatani method, frequency-resonance diagnostics.
}
\par \textbf{\textit{Keywords}—non-invasive diagnostics, artificial intelligence, diagnostic decision support system.}
\begin{center}
    \RomanNumeralCaps{1}. Introduction
\end{center}
\par Health is the most valuable resource of the state. One
of the task of modern society is to timely detect the
disease risk. The implementation of this task requires
new diagnostic tools based on the latest technologies.
Risk diagnosis will provide significant economics savings
towards disease prevention and treatment, as well as
improve the quality of primary health care. Risk is the
probability of developing a disease [1].
\par The current problem in the area of risk diagnostic is
the creation of non-invasive technology for examination
and detection of diseases at early stages in order to carry
out individualized prevention. The emerging modern
technologies provide ample opportunities for solving this
problem [2]. At the same time, let us quote a doctor’s
critical statements about informatization of medicine:
"The global problem is the lack of resources. And we
are not talking about the shortage of money, but about
the shortage of time. The time of professionals is the
main world deficit. Information technology offers great
opportunities to save money. Telemedicine, for example,
has a huge potential. Support for medical decision
making is of enormous value, but it is not being deployed
and practiced" [3].
\par The importance of the intelligent non-invasive diagnostics problem has several aspects. Firstly, it is \textbf{caring}
for people’s health that leads towards the individual
health improvement and preventive care. Secondly, it
is an \textbf{increase in the quality} of individual preventive
care to the population. Thirdly, it is \textbf{beneficial} from
the economic point of view, as the costs of prevention
and treatment are minimized. Taking into account the
problem of "time shortage", it is important to minimize
time costs, as the procedures are carried out quickly
enough. It is important that non-invasive diagnostics
procedures are safe and painless.
\par Therefore, there is a need to continue to investigate
and develop the intelligent non-invasive diagnostics, with
the primary focus on the development of an intelligent
system to support the decision making for non-invasive
diagnostic.
\par The proposed architecture for the intelligent diagnostic
ostis-system allows to assess the risk of diseases in
patients, and creates the "windows of opportunity" not
only for patients and doctors, but also for developers in
terms of expanding the functionality of the system.
\par The main aim of this paper is to create an intelligent
non-invasive diagnostic system architecture suitable for
screening of systemic and nosological risks and early
diagnosis of diseases, i. e. for diagnosing latent and
initial stages of pathological process development for the
purpose of primary and secondary prevention or timely
treatment.
\begin{center}
    \RomanNumeralCaps{2}. Overview of Existing Solutions
\end{center}
\par The quality of medical care depends on the level of
doctors’ training and on systems that support decision making, including in the field of diagnosing the diseases
at various stages.
\par While there are many medical decision support systems in various fields, the deployment of such systems
into the everyday practice is relatively slow.
\end{multicols}
\end{document}
